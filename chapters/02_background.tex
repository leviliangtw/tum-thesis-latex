% !TeX root = ../main.tex
% Add the above to each chapter to make compiling the PDF easier in some editors.

\chapter{Background}\label{chapter:background}

\section{System Monitoring}

\subsection{Black Box Monitoring}

Black box monitoring is a well-established approach to system analysis, most commonly applied in software testing and network monitoring. It evaluates a system's functionality based on its responses to given inputs while abstracting away the system's internal workings. This technique metaphorically views the system as a "black box", signifying that the internal mechanisms are not visible or accounted for in the evaluation process~\parencite{beizer_black_1996}~\parencite{hierons_software_2006}.

One of the key advantages of black box monitoring is its simplicity and ease of implementation~\parencite{jain_art_1991}. Since it doesn't require in-depth knowledge of the system internals, it can be quickly deployed across various platforms and systems. This quality fosters wide applicability, accommodating diverse programming languages and system architectures. Furthermore, black box monitoring has a unique focus on the user experience. By simulating user actions and recording the system's responses, it offers a valuable measure of system functionality from a user's perspective~\parencite{jorgensen_software_2021}. Thus, it is instrumental in ensuring that the system effectively meets the end-user's requirements. However, while the black box approach has its merits, it is not without challenges. One significant limitation is the potential for limited coverage. Since black box monitoring concentrates on the inputs and outputs, it might overlook internal system issues~\parencite{kaner_testing_1999}. If a problem doesn't immediately or directly affect the system output, it may remain undetected, only to potentially cause serious issues down the line. Moreover, black box monitoring could contribute to inefficiency in troubleshooting. A system failure detected by black box testing can be difficult to diagnose and fix due to the lack of visibility into the system internals. Pinpointing the specific area of the system causing the issue may turn into a challenging endeavor~\parencite{kaner_testing_1999}.

Despite these challenges, the future of black box monitoring looks promising, particularly in the context of our rapidly digitalizing world. Given the rising importance of customer satisfaction and user experience in the digital economy, black box monitoring's ability to capture system performance from a user's viewpoint remains invaluable. The evolution of \ac{AI} and \ac{ML} technologies could significantly enhance the efficiency of black box monitoring. AI-powered monitoring tools can automate the process of generating test inputs and predicting system outputs, bringing about a higher degree of efficiency and reliability~\parencite{bertolino_software_2003}. As the IT landscape becomes more complex with microservices and cloud-based applications, understanding the internals of every service can be overwhelming. In such a scenario, black box monitoring's ability to provide a holistic view of system behavior can be highly beneficial. Moreover, in light of growing privacy regulations and data protection measures, black box monitoring's non-invasive approach aligns with the current trends. Focusing on system outputs rather than internals could potentially minimize privacy or data protection concerns~\parencite{otto_addressing_2007}.

In conclusion, black box monitoring remains a vital tool for system testing and monitoring, despite its inherent challenges. Its future appears increasingly integrated with AI technologies and aligned with user-centric design principles. Nevertheless, black box monitoring should not be considered a stand-alone solution but rather a component of a comprehensive monitoring strategy that incorporates various techniques to effectively manage system performance.