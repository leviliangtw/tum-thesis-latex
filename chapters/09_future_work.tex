% !TeX root = ../main.tex
% Add the above to each chapter to make compiling the PDF easier in some editors.

\chapter{Future Work}\label{chapter:future_work}

No matter how IT industries evolve, advancing system monitoring techniques cannot be overlooked. The thesis has laid the groundwork for understanding and enhancing the observability of complex IT infrastructures with Active Monitoring, emphasizing utilizing the Prometheus Operator and Blackbox Exporter. Nonetheless, a vast expanse of uncharted territory remains ripe for exploration. The following paragraphs introduce promising directions for future research, each offering the potential to significantly refine and advance our current monitoring capabilities. These directions include the automation of Blackbox Exporter configurations, developing interfaces for custom probers, and integrating eBPF programs for low-cost, high-precision monitoring. 

A promising area for further investigation involves enhancing the Prometheus Operator's functionality, particularly in managing the Blackbox Exporter's configuration through \ac{CRD}s. Automating this aspect would mark a significant leap toward more manageable and efficient monitoring systems, reducing the manual overhead for maintainers and aligning the Blackbox Exporter more closely with the automated, operator-managed setup of the Prometheus Agent. This evolution toward automated configurations promises to elevate the efficiency of monitoring operations and ensure a higher degree of flexibility and ease of maintenance. 

Exploring an interface for creating and managing custom probers within the Blackbox Exporter framework represents another direction for research. The limitation to officially supported probers restricts the tool's applicability across diverse monitoring scenarios. By developing a customizable interface, users could employ monitoring solutions to their specific needs, significantly enhancing the tool's versatility and the relevance of the data it collects. This advancement would enable more delicate monitoring approaches, allowing for a broader application spectrum and more targeted data collection strategies. 

Furthermore, the potential integration of eBPF programs into the Blackbox Exporter offers an exciting prospect for the future of monitoring. eBPF, known as Extended Berkeley Packet Filter, is a technology that allows efficient performance monitoring within the kernel space without requiring traditional overheads. Leveraging eBPF's ability to perform low-level, low-cost analyses could transform the observability landscape of Active Monitoring, providing insights into system performance with minimal overhead. This approach could lead to more effective and efficient monitoring strategies, enabling a deeper understanding of system behaviors and facilitating more fine-grained performance improvement. 

Beyond these specific areas, the continuous evolution of cloud computing and container technology presents ongoing opportunities for the refinement of monitoring strategies and tools. The integration of advanced machine learning techniques for predictive monitoring, the development of finer-grained tools for microservices architectures, and the investigation of automated system recovery mechanisms based on monitoring data are critical areas where future research could drive significant improvements. By pushing the boundaries in these directions, the field can anticipate and adapt to the changing landscape of IT infrastructures, ensuring that monitoring solutions remain robust, effective, and aligned with the needs of modern technologies. 

In sum, the future work mentioned here is not just an extension of this thesis but a roadmap for advancing system monitoring into new realms of efficiency, flexibility, and observability. By pursuing these research directions, the field can look forward to monitoring solutions that are not only more capable but also more adapted to the dynamic complexities of modern IT environments. 