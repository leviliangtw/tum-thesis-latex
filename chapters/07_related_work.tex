% !TeX root = ../main.tex
% Add the above to each chapter to make compiling the PDF easier in some editors.

\chapter{Related Work}\label{chapter:related_work}

Modern IT infrastructure management, monitoring, and analytics platforms ensure system reliability and performance. Some industries design and customize their solution, like Amadeus, while some employ mature existing commercial solutions. Among these solutions, as a representative player, Datadog is known for its robust functionalities and capabilities. Therefore, the upcoming section will introduce two solutions for active monitoring: the solution developed and used internally by Amadeus since 2021 and another solution driven by Datadog's synthetic monitoring~\parencite{datadogSyntheticMonitoring}. 

\section{Amadeus' Active Monitoring}

\begin{figure}[htpb]
  \centering
  % This should probably go into a file in figures/
  \begin{tikzpicture}[
      shorten >=1pt,
      zonetag/.style={align=center, font=\fontsize{10}{10}\color{black!60}\ttfamily, anchor=west},
      node/.style={draw, align=center, minimum height=3em, anchor=west, rounded corners}, 
      zone/.style={draw=black!60, inner sep=8pt, anchor=base}
    ]

    \node (user) [node distance=5em, text width=3em, node, circle] {User};

    \node (operator) [node distance=3em, text width=8em, node, below=of user] {Operator\\(Handle CR)};
    \node (scheduler) [node distance=3em, text width=8em, node, below=of operator] {Scheduler\\(Manage Probers)};
    \node (prober1) [node distance=3em, text width=8em, node, below=of scheduler] {HTTP Prober};
    \node (prober2) [node distance=3em, text width=8em, node, left=of prober1] {TCP Prober};
    \node (prober3) [node distance=3em, text width=8em, node, right=of prober1] {HTTPS Prober};
    \node (target1) [node distance=3em, text width=8em, node, below=of prober1] {Target B};
    \node (target2) [node distance=3em, text width=8em, node, left=of target1] {Target A};
    \node (target3) [node distance=3em, text width=8em, node, right=of target1] {Target C};

    \node (CZ) [zone, fit={(operator) (scheduler) (prober1) (prober2) (prober3) (target1) (target2) (target3)}] {};
    \node (C)[node distance=6pt, above=of CZ.north west, zonetag] {OpenShift Cluster};

    \path[->] (user) edge node[auto, midway] {Create CR} (operator);
    \path[->] (operator) edge node[auto] {Configure} (scheduler);
    \path[->] (scheduler) edge node[auto] {Trigger} (prober1);
    \path[->] (scheduler) edge node[auto] {Trigger} (prober2);
    \path[->] (scheduler) edge node[auto] {Trigger} (prober3);
    \path[->] (prober1) edge node[auto] {} (target1);
    \path[->] (prober2) edge node[auto] {Probe} (target1);
    \path[->] (prober3) edge node[auto] {Probe} (target3);
    \path[->] (prober3) edge node[auto] {Probe} (target2);
  \end{tikzpicture}
  \caption[Amadeus' Active Monitoring]{Amadeus' Active Monitoring.}\label{fig:amadeus-active-monitoring}
\end{figure}

Inspired by the Blackbox Monitoring principles, Amadeus developed the active monitoring system by integrating the Operator pattern, aiming for a simple cloud monitoring solution. As shown in \autoref{fig:amadeus-active-monitoring}, this system is architecturally composed of three pivotal elements: the operator, the scheduler, and the prober. This design facilitates a lightweight yet practical approach to cloud-native monitoring. 

Users initially tailor their monitoring setup using \ac{CR}. The operator is key in interpreting these \ac{CR}s to establish and maintain the scheduler's monitoring configuration. This approach highlights the system's flexibility, allowing users to define monitoring configurations that meet their specific needs. 

The scheduler is central to the system's functionality, orchestrating the monitoring process. It is responsible for initiating monitoring requests, per the configurations detailed in the \ac{CR}s, and coordinating with the probers to gather the requisite data. The probers, designed as workers, are tasked with executing the monitoring by probing the targeted services and directing the collected data back to the scheduler. This component-driven approach ensures a comprehensive monitoring cycle. 

Despite its sound design, the system's limitations restrain its potency, particularly in large-scale applications. Notably, the architecture presents a single point of failure at the scheduler, posing risks of data collection interruptions and potential traffic bottlenecks. The lack of scalability poses these challenges, making it difficult to efficiently monitor a vast array of distributed services across multiple clusters. 

\section{Datadog's Synthetic Monitoring}

\begin{figure}[htpb]
  \centering
  % This should probably go into a file in figures/
  \begin{tikzpicture}[
      shorten >=1pt,
      zonetag/.style={align=center},
      node/.style={draw, align=center, minimum height=3em, anchor=west, rounded corners}, 
      zone/.style={draw=black!60, inner sep=8pt, anchor=base}
    ]

    \node (datadog) [node distance=5em, text width=6em, node] {Datadog};

    \node (agent1) [node distance=4em, text width=4em, node, below left=of datadog] {Agent};
    \node (target1) [node distance=3em, text width=4em, node, left=of agent1] {Target};
    \node (target2) [node distance=2em, text width=4em, node, below=of target1] {Target};
    \node (target3) [node distance=2em, text width=4em, node, below=of target2] {Target};
    \node (CZ1) [zone, fit={(agent1) (target1) (target2) (target3)}] {};
    \draw[->] (datadog) -| node[auto, left] {Pull Metrics} (agent1);
    \draw[->] (agent1) edge node[auto, below] {} (target1);
    \draw[->] (agent1) |- node[auto, right] {API Test} (target2);
    \draw[->] (agent1) |- node[auto] {} (target3);

    \node (agent2) [node distance=4em, text width=4em, node, below right=of datadog] {Agent};
    \node (target4) [node distance=3em, text width=4em, node, right=of agent2] {Target};
    \node (target5) [node distance=2em, text width=4em, node, below=of target4] {Target};
    \node (target6) [node distance=2em, text width=4em, node, below=of target5] {Target};
    \node (CZ2) [zone, fit={(agent2) (target4) (target5) (target6)}] {};
    \draw[->] (datadog) -| node[auto, right]{Pull Metrics} (agent2);
    \draw[->] (agent2) edge node[auto, below] {} (target4);
    \draw[->] (agent2) |- node[auto, left] {API Test} (target5);
    \draw[->] (agent2) |- node[auto] {} (target6);

    \draw[-] (agent1) edge node[auto] { Internal Network} (agent2);

  \end{tikzpicture}
  \caption[Datadog's Synthetic Monitoring]{Datadog's Synthetic Monitoring.}\label{fig:datadog-synthetic-monitoring}
\end{figure}

% In the realm of modern IT infrastructure management, monitoring and analytics platforms play an integral role in ensuring system reliability and performance. Among the existing solutions, Datadog has emerged as a prominent player, known for its robust functionalities and capabilities~\parencite{datadogSyntheticMonitoring}. In the context of this project, which aims to design a dedicated and distributed active monitoring system, understanding Datadog's approach and technology is pivotal. 

As an integrated platform for monitoring, Datadog collects and analyzes data from multiple sources, offering a unified view of an organization's IT infrastructure. A standout feature of Datadog is the Synthetic Monitoring (see \autoref{fig:datadog-synthetic-monitoring}), which is particularly applicable for active monitoring~\parencite{datadogSyntheticMonitoring} [Data]. This feature simulates requests and actions, thereby offering insights into the performance of APIs across various network layers, from backend to frontend~\parencite{datadogAPITests}~\parencite{datadogBrowserTests}. 

There are two distinct elements of Datadog's Synthetic Monitoring that are of particular interest: API Tests and Browser Tests~\parencite{datadogAPITests}~\parencite{datadogBrowserTests}. API Tests actively probe target services, gathering their statuses and metrics. These tests offer a snapshot of the availability of the monitored services, a crucial factor in any active monitoring system. In Datadog's implementation, API Tests form the bulk of the monitoring process, thereby highlighting their importance. Browser Tests, on the other hand, execute defined scenarios on target services using a chosen browser. This essentially reproduces the actions of a user, providing experiential feedback about the services. This user-centric approach complements the more technical data gathered by the API Tests, contributing to a more holistic view of the system's health. 

When it comes to monitoring availability, it is crucial to decide the position of source endpoints that proceed with monitoring~\parencite{datadogRunSyntheticTests}. Datadog provides the feature to customize private locations to execute Synthetic Monitoring. By installing Docker containers as private endpoints in desired locations, Synthetic Monitoring, including API Tests and Browser Tests, can employ these remote endpoints as its starting point. 

While Datadog's synthetic monitoring approach presents a well-rounded solution, it operates primarily within a centralized model in terms of the agents' side. To elaborate, monitoring targets from a private location requires an installed agent in the same cluster, and the agent could be a single point of failure. 