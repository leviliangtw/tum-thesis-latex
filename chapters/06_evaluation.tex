% !TeX root = ../main.tex
% Add the above to each chapter to make compiling the PDF easier in some editors.

\chapter{Evaluation}\label{chapter:evaluation}

This evaluation study seeks to determine if the design leads to a tolerable overhead compared to the original in-house monitoring application that Amadeus developed. Both applications operate on a private OpenShift Cluster managed by the Amadeus \ac{SRE} team. The primary metrics collected include Openshift Metrics for \ac{CPU}, memory, and network, gathered using cAdvisor or the Network Metrics Daemon~\parencite{redhatAssociatingSecondaryInterfaces}~\parencite{redhatPrometheusClusterMonitoring}. Subsequent sections will outline the evaluation methodology and present the findings from the system analysis.


\section{Methodology}

The evaluation focuses on analyzing system behaviors concerning \ac{CPU} utilization, memory utilization, and network traffic. These three entities are critical for developers to gain insights into system overhead and performance intuitively. Understanding how a system allocates and utilizes its resources, along with how it communicates internally and externally, provides a comprehensive view of its efficiency and scalability.

\subsection{\ac{CPU} Utilization}

For \ac{CPU} utilization analysis within a pod, the metric "container\_cpu\_usage\_seconds\_total" will be utilized. This metric measures the cumulative \ac{CPU} time consumed by a container in seconds. The methodology involves:

\begin{itemize}
    \item Collecting \ac{CPU} usage data over time to understand the baseline and peak \ac{CPU} utilization patterns.
    \item Analyzing the \ac{CPU} consumption in correlation with different applications to identify any potential inefficiencies.
    \item Comparing \ac{CPU} utilization across different applications to assess resource allocation effectiveness.
\end{itemize}

This analysis will help in understanding the \ac{CPU} demands of the applications running in OpenShift Pods and how effectively \ac{CPU} resources are utilized.

\subsection{Memory Utilization}

Memory utilization will be assessed using the "container\_memory\_working\_set\_bytes" metric, which provides the amount of memory actively used by a container, excluding unused pages. The evaluation methodology includes:

\begin{itemize}
    \item Monitoring the working set memory to identify memory usage under different operational conditions.
    \item Investigating memory patterns to ensure that applications operate appropriately and efficiently use memory resources.
    \item Evaluating memory usage trends over time to evaluate resource requirements. 
\end{itemize}

This evaluation aims to highlight memory utilization efficiency and potential areas for optimization within the OpenShift environment.

\subsection{Network I/O}

Network I/O will be analyzed through "container\_network\_receive\_bytes\_total" and "container\_network\_transmit\_bytes\_total" metrics, representing the total bytes received and transmitted by the container network interface, respectively. The methodology involves:

\begin{itemize}
    \item Monitoring inbound and outbound network traffic to identify communication patterns.
    \item Assessing network traffic volume in relation to application activity.
    \item Analyzing network traffic trends to assess data flow efficiency.
\end{itemize}

By examining network I/O, we aim to understand the network performance and efficiency of applications running in OpenShift Pods, highlighting areas for network optimization.

\section{System Analysis}

Prometheus' and Amadeus's solutions feature a similar architecture: the scheduler triggers the prober to carry out monitoring tasks while the prober probes targets. The forthcoming analysis will span 12 hours, focusing on the average hourly values of each metric. Given that the average target count in production environments is nearing 100, the following analysis will address both the scheduler and the prober within the scenario of probing 100 targets. 

\subsection{Scheduler Analysis}

\begin{figure}[htpb]
    \centering
    \pgfplotstableset{col sep=&, row sep=\\}
    % This should probably go into a file in data/
    \pgfplotstableread{
      a & b    \\
      1 & 4.27 \\
      2 & 4.19 \\
      3 & 4.16 \\
      4 & 4.18 \\
      5 & 4.23 \\
      6 & 4.23 \\
      7 & 4.27 \\
      8 & 4.23 \\
      9 & 4.19 \\
      10 & 4.23 \\
      11 & 4.21 \\
      12 & 4.25 \\
    }\cpuA
    \pgfplotstableread{
      a & b    \\
      1 & 2.32 \\
      2 & 2.31 \\
      3 & 2.30 \\
      4 & 2.49 \\
      5 & 2.49 \\
      6 & 2.56 \\
      7 & 2.71 \\
      8 & 2.62 \\
      9 & 2.64 \\
      10 & 2.60 \\
      11 & 2.58 \\
      12 & 2.62 \\
    }\cpuB
    \pgfplotstableread{
      a & b    \\
      1 & 123 \\
      2 & 124 \\
      3 & 123 \\
      4 & 123 \\
      5 & 123 \\
      6 & 124 \\
      7 & 124 \\
      8 & 124 \\
      9 & 125 \\
      10 & 124 \\
      11 & 125 \\
      12 & 125 \\
    }\memA
    \pgfplotstableread{
      a & b    \\
      1 & 267 \\
      2 & 267 \\
      3 & 267 \\
      4 & 267 \\
      5 & 267 \\
      6 & 267 \\
      7 & 267 \\
      8 & 267 \\
      9 & 267 \\
      10 & 267 \\
      11 & 267 \\
      12 & 267 \\
    }\memB
    \pgfplotstableread{
      a & b    \\
      1 & 3.26 \\
      2 & 3.26 \\
      3 & 3.25 \\
      4 & 3.25 \\
      5 & 3.26 \\
      6 & 3.25 \\
      7 & 3.25 \\
      8 & 3.26 \\
      9 & 3.26 \\
      10 & 3.25 \\
      11 & 3.27 \\
      12 & 3.26 \\
    }\netinA
    \pgfplotstableread{
      a & b    \\
      1 & 2.04 \\
      2 & 2.03 \\
      3 & 2.03 \\
      4 & 2.04 \\
      5 & 2.06 \\
      6 & 2.05 \\
      7 & 2.04 \\
      8 & 2.04 \\
      9 & 2.04 \\
      10 & 2.05 \\
      11 & 2.06 \\
      12 & 2.04 \\
    }\netinB
    \pgfplotstableread{
      a & b    \\
      1 & 3.28 \\
      2 & 3.28 \\
      3 & 3.27 \\
      4 & 3.27 \\
      5 & 3.28 \\
      6 & 3.27 \\
      7 & 3.27 \\
      8 & 3.28 \\
      9 & 3.27 \\
      10 & 3.27 \\
      11 & 3.29 \\
      12 & 3.27 \\
    }\netoutA
    \pgfplotstableread{
      a & b    \\
      1 & 9.51 \\
      2 & 9.50 \\
      3 & 9.39 \\
      4 & 9.55 \\
      5 & 9.55 \\
      6 & 9.53 \\
      7 & 9.49 \\
      8 & 9.54 \\
      9 & 9.54 \\
      10 & 9.57 \\
      11 & 9.58 \\
      12 & 9.52 \\
    }\netoutB
    % This should probably go into a file in figures/
    \scalebox{.85}{\begin{tabular}{ c c }
        \begin{tikzpicture}
            \begin{axis}[
              ymin=0,
              grid,
              thick,
              ylabel=CPU (millicore),
              xlabel=Time (hour),
              legend columns=-1,
              legend style={legend pos=south east},
              legend cell align={left}, 
              legend to name=scheduler-analysis
              ]
              \addplot[mark=*, blue] table[x=a, y=b]{\cpuA};
              \addplot[mark=x, red] table[x=a, y=b]{\cpuB};
              \addlegendentry{Prometheus Agent\ \ \ \ };
              \addlegendentry{Amadeus' Scheduler};
            \end{axis}
        \end{tikzpicture} &
        \begin{tikzpicture}
            \begin{axis}[
                ymin=0,
                grid,
                thick,
                ylabel=Memory (MB),
                xlabel=Time (hour)
              ]
              \addplot[mark=*, blue] table[x=a, y=b]{\memA};
              \addplot[mark=x, red] table[x=a, y=b]{\memB};
            \end{axis}
        \end{tikzpicture} \\ 
        \begin{tikzpicture}
            \begin{axis}[
                ymin=0,
                grid,
                thick,
                ylabel=Network In (kB/s),
                xlabel=Time (hour)
              ]
              \addplot[mark=*, blue] table[x=a, y=b]{\netinA};
              \addplot[mark=x, red] table[x=a, y=b]{\netinB};
            \end{axis}
          \end{tikzpicture} &
          \begin{tikzpicture}
            \begin{axis}[
                ymin=0,
                grid,
                thick,
                ylabel=Network Out (kB/s),
                xlabel=Time (hour)
              ]
              \addplot[mark=*, blue] table[x=a, y=b]{\netoutA};
              \addplot[mark=x, red] table[x=a, y=b]{\netoutB};
            \end{axis}
          \end{tikzpicture}  
    \end{tabular}}
    \scalebox{.85}{\ref{scheduler-analysis}}
    \caption[Scheduler Analysis]{Scheduler analysis.}\label{fig:scheduler-analysis}
\end{figure}

The Prometheus Agent and Amadeus' Scheduler perform similar roles but differ in their implementations, with the Prometheus Agent providing additional features. As depicted in the \autoref{fig:scheduler-analysis}, the CPU utilization between the two schedulers shows slight differences due to the minor units in the measurement. Surprisingly, memory utilization by the Prometheus Agent was significantly reduced by approximately 140 MB. Regarding network I/O, given that the measurements are in kB, the differences are relatively inconsequential. Adopting the Prometheus Agent results in an acceptable overhead while substantially decreasing memory consumption. 

\subsection{Prober Analysis}

\begin{figure}[htpb]
  \centering
  \pgfplotstableset{col sep=&, row sep=\\}
  % This should probably go into a file in data/
  \pgfplotstableread{
    a & b    \\
    1 & 16.6 \\
    2 & 16.7 \\
    3 & 16.8 \\
    4 & 16.9 \\
    5 & 16.8 \\
    6 & 16.9 \\
    7 & 16.7 \\
    8 & 16.8 \\
    9 & 17.1 \\
    10 & 16.9 \\
    11 & 17.0 \\
    12 & 16.9 \\
  }\cpuA
  \pgfplotstableread{
    a & b    \\
    1 & 5.82 \\
    2 & 5.92 \\
    3 & 5.87 \\
    4 & 5.91 \\
    5 & 5.91 \\
    6 & 5.89 \\
    7 & 5.91 \\
    8 & 5.95 \\
    9 & 5.99 \\
    10 & 5.84 \\
    11 & 5.97 \\
    12 & 5.94 \\
  }\cpuB
  \pgfplotstableread{
    a & b    \\
    1 & 43.6 \\
    2 & 43.4 \\
    3 & 43.4 \\
    4 & 43.3 \\
    5 & 43.5 \\
    6 & 43.5 \\
    7 & 43.5 \\
    8 & 43.5 \\
    9 & 44.0 \\
    10 & 43.8 \\
    11 & 44.0 \\
    12 & 44.1 \\
  }\memA
  \pgfplotstableread{
    a & b    \\
    1 & 297 \\
    2 & 297 \\
    3 & 297 \\
    4 & 297 \\
    5 & 297 \\
    6 & 297 \\
    7 & 297 \\
    8 & 297 \\
    9 & 297 \\
    10 & 297 \\
    11 & 297 \\
    12 & 297 \\
  }\memB
  \pgfplotstableread{
    a & b    \\
    1 & 30.1 \\
    2 & 30.1 \\
    3 & 30.1 \\
    4 & 30.1 \\
    5 & 30.0 \\
    6 & 30.1 \\
    7 & 30.1 \\
    8 & 30.2 \\
    9 & 30.2 \\
    10 & 30.1 \\
    11 & 30.1 \\
    12 & 30.2 \\
  }\netinA
  \pgfplotstableread{
    a & b    \\
    1 & 13.5 \\
    2 & 13.5 \\
    3 & 13.5 \\
    4 & 13.5 \\
    5 & 13.5 \\
    6 & 13.5 \\
    7 & 13.3 \\
    8 & 13.3 \\
    9 & 13.5 \\
    10 & 13.5 \\
    11 & 13.5 \\
    12 & 13.5 \\
  }\netinB
  \pgfplotstableread{
    a & b    \\
    1 & 11.2 \\
    2 & 11.2 \\
    3 & 11.3 \\
    4 & 11.3 \\
    5 & 11.2 \\
    6 & 11.3 \\
    7 & 11.3 \\
    8 & 11.3 \\
    9 & 11.3 \\
    10 & 11.3 \\
    11 & 11.3 \\
    12 & 11.3 \\
  }\netoutA
  \pgfplotstableread{
    a & b    \\
    1 & 9.79 \\
    2 & 9.78 \\
    3 & 9.76 \\
    4 & 9.78 \\
    5 & 9.78 \\
    6 & 9.79 \\
    7 & 9.61 \\
    8 & 9.62 \\
    9 & 9.78 \\
    10 & 9.77 \\
    11 & 9.78 \\
    12 & 9.76 \\
  }\netoutB
  % This should probably go into a file in figures/
  \scalebox{.85}{\begin{tabular}{ c c }
      \begin{tikzpicture}
          \begin{axis}[
              ymin=0,
              grid,
              thick,
              ylabel=CPU (millicore),
              xlabel=Time (hour),
              legend columns=-1,
              legend style={legend pos=south east},
              legend cell align={left}, 
              legend to name=prober-analysis,
            ]
            \addplot[mark=*, blue] table[x=a, y=b]{\cpuA};
            \addplot[mark=x, red] table[x=a, y=b]{\cpuB};
            \addlegendentry{Blackbox Exporter\ \ \ \ };
            \addlegendentry{Amadeus' HTTP Prober};
          \end{axis}
      \end{tikzpicture} &
      \begin{tikzpicture}
          \begin{axis}[
              ymin=0,
              grid,
              thick,
              ylabel=Memory (MB),
              xlabel=Time (hour),
            ]
            \addplot[mark=*, blue] table[x=a, y=b]{\memA};
            \addplot[mark=x, red] table[x=a, y=b]{\memB};
          \end{axis}
      \end{tikzpicture} \\ 
      \begin{tikzpicture}
          \begin{axis}[
              ymin=0,
              legend style={legend pos=south east},
              grid,
              thick,
              ylabel=Network In (kB/s),
              xlabel=Time (hour),
              legend cell align={left}
            ]
            \addplot[mark=*, blue] table[x=a, y=b]{\netinA};
            \addplot[mark=x, red] table[x=a, y=b]{\netinB};
          \end{axis}
        \end{tikzpicture} &
      \begin{tikzpicture}
        \begin{axis}[
            ymin=0,
            grid,
            thick,
            ylabel=Network Out (kB/s),
            xlabel=Time (hour),
          ]
          \addplot[mark=*, blue] table[x=a, y=b]{\netoutA};
          \addplot[mark=x, red] table[x=a, y=b]{\netoutB};
        \end{axis}
      \end{tikzpicture}
  \end{tabular}}
  \scalebox{.85}{\ref{prober-analysis}}
  \caption[Prober Analysis]{Prober analysis.}\label{fig:prober-analysis}
\end{figure}

The Blackbox Exporter, capable of conducting customized probes across different protocols, contrasts with Amadeus' HTTP Prober's limitation to only HTTP probes. For a fair comparison, only the HTTP module is used in the Blackbox Exporter. \autoref{fig:prober-analysis} shows the Blackbox Exporter's CPU usage is slightly higher than Amadeus' by about 10 millicores, a minor increase justified by its extended features and considered acceptable. Notably, it also offers a substantial memory usage improvement of approximately 250 MB. Although its network input is double that of its counterpart, investigations reveal this is due to the Prometheus Agent's more detailed HTTP GET requests, which are not expected to cause issues, even when scaling up to 10 or 100 times the number of targets due to the lightweight nature of these requests. In summary, the minor increases in CPU and network input by the Blackbox Exporter are more than compensated for by considerable memory savings, making it a fully acceptable solution. 

\section{Overhead Analysis}

This overhead analysis evaluates the escalating load rate as the target number increases from 100 to 1000. Utilizing these findings could enhance the use of \ac{HPA} by identifying appropriate threshold metrics. All metrics were recorded as hourly averages for each target number, expected to increase proportionally with the target number in theory: 

\begin{figure}[htpb]
  \centering
  \pgfplotstableset{col sep=&, row sep=\\}
  % This should probably go into a file in data/
  \pgfplotstableread{
    a & b    \\
    100 & 4.13 \\
    200 & 5.46 \\
    300 & 6.95 \\
    400 & 10.0 \\
    500 & 11.7 \\
    600 & 13.2 \\
    700 & 14.3 \\
    800 & 16.1 \\
    900 & 20.2 \\
    1000 & 22.0 \\
  }\cpuA
  \pgfplotstableread{
    a & b    \\
    100 & 16.8 \\
    200 & 32.7 \\
    300 & 48.6 \\
    400 & 63.8 \\
    500 & 79.7 \\
    600 & 95.9 \\
    700 & 112 \\
    800 & 128 \\
    900 & 144 \\
    1000 & 160 \\
  }\cpuB
  \pgfplotstableread{
    a & b    \\
    100 & 245 \\
    200 & 328 \\
    300 & 426 \\
    400 & 496 \\
    500 & 610 \\
    600 & 720 \\
    700 & 766 \\
    800 & 836 \\
    900 & 919 \\
    1000 & 1075 \\
  }\memA
  \pgfplotstableread{
    a & b    \\
    100 & 92.7 \\
    200 & 101 \\
    300 & 110 \\
    400 & 124 \\
    500 & 133 \\
    600 & 145 \\
    700 & 155 \\
    800 & 191 \\
    900 & 200 \\
    1000 & 209 \\
  }\memB
  \pgfplotstableread{
    a & b    \\
    100 & 3.25 \\
    200 & 6.31 \\
    300 & 9.36 \\
    400 & 12.5 \\
    500 & 15.5 \\
    600 & 18.6 \\
    700 & 21.7 \\
    800 & 24.7 \\
    900 & 27.7 \\
    1000 & 30.8 \\
  }\netinA
  \pgfplotstableread{
    a & b    \\
    100 & 30.3 \\
    200 & 60.5 \\
    300 & 90.6 \\
    400 & 121 \\
    500 & 150 \\
    600 & 182 \\
    700 & 211 \\
    800 & 241 \\
    900 & 272 \\
    1000 & 302 \\
  }\netinB
  \pgfplotstableread{
    a & b    \\
    100 & 3.29 \\
    200 & 6.34 \\
    300 & 9.35 \\
    400 & 12.4 \\
    500 & 15.4 \\
    600 & 18.5 \\
    700 & 21.5 \\
    800 & 24.5 \\
    900 & 27.5 \\
    1000 & 30.5 \\
  }\netoutA
  \pgfplotstableread{
    a & b    \\
    100 & 11.6 \\
    200 & 19.1 \\
    300 & 26.6 \\
    400 & 33.8 \\
    500 & 41.4 \\
    600 & 49.2 \\
    700 & 56.6 \\
    800 & 64.2 \\
    900 & 71.8 \\
    1000 & 79.4 \\
  }\netoutB
  % This should probably go into a file in figures/
  \scalebox{.85}{\begin{tabular}{ c c }
      \begin{tikzpicture}
          \begin{axis}[
            ymin=0,
            grid,
            thick,
            ylabel=CPU (millicore),
            xlabel=Target Number,
            legend columns=-1,
            legend style={legend pos=south east},
            legend cell align={left}, 
            legend to name=overhead-analysis
            ]
            \addplot[mark=*, blue] table[x=a, y=b]{\cpuA};
            \addplot[mark=x, red] table[x=a, y=b]{\cpuB};
            \addlegendentry{Prometheus Agent\ \ \ \ };
            \addlegendentry{Blackbox Exporter};
          \end{axis}
      \end{tikzpicture} &
      \begin{tikzpicture}
          \begin{axis}[
              ymin=0,
              grid,
              thick,
              ylabel=Memory (MB),
              xlabel=Target Number
            ]
            \addplot[mark=*, blue] table[x=a, y=b]{\memA};
            \addplot[mark=x, red] table[x=a, y=b]{\memB};
          \end{axis}
      \end{tikzpicture} \\ 
      \begin{tikzpicture}
          \begin{axis}[
              ymin=0,
              grid,
              thick,
              ylabel=Network In (kB/s),
              xlabel=Target Number
            ]
            \addplot[mark=*, blue] table[x=a, y=b]{\netinA};
            \addplot[mark=x, red] table[x=a, y=b]{\netinB};
          \end{axis}
        \end{tikzpicture} &
        \begin{tikzpicture}
          \begin{axis}[
              ymin=0,
              grid,
              thick,
              ylabel=Network Out (kB/s),
              xlabel=Target Number
            ]
            \addplot[mark=*, blue] table[x=a, y=b]{\netoutA};
            \addplot[mark=x, red] table[x=a, y=b]{\netoutB};
          \end{axis}
        \end{tikzpicture}  
  \end{tabular}}
  \scalebox{.85}{\ref{overhead-analysis}}
  \caption[Overhead Analysis]{Overhead analysis.}\label{fig:overhead-analysis}
\end{figure}

As shown in the above \autoref{fig:overhead-analysis}, all metrics demonstrate a consistent linear growth trend with each addition of 100 targets. Initially, the Prometheus Agent exhibits significant memory usage and a higher rate of increase, attributed to its role in scheduling probes, which involves managing numerous scraping jobs and their associated configurations. Therefore, utilizing the memory metrics from the Prometheus Agent for \ac{HPA} is a reasonable choice. Conversely, the behavior of the Blackbox Exporter differs from the Prometheus Agent. It displays a modest upward trend in memory usage alongside more pronounced CPU and network utilization increases. Notably, the growth rates for CPU and network usage are comparable, indicating that the overhead is primarily relevant to network-related tasks. Thus, CPU metrics are suitable as threshold indicators for the Blackbox Exporter. 

\section{Scraping Analysis}

This analysis assesses the efficiency of the Prometheus Agent's scraping approach by tracking the "Network In" metric of the Prober, which can be regarded as an indicator of the traffic generated by the Scheduler's probing requests. The traffic is quantified as the average per minute, with the following figure metrics over 30 minutes: 

\begin{figure}[htpb]
  \centering
  \pgfplotstableset{col sep=&, row sep=\\}
  % This should probably go into a file in data/
  \pgfplotstableread{
    a & b    \\
    1 & 33.7 \\
    2 & 39.8 \\
    3 & 30.8 \\
    4 & 27.1 \\
    5 & 28.4 \\
    6 & 32.1 \\
    7 & 32.1 \\
    8 & 32.3 \\
    9 & 25.4 \\
    10 & 19.3 \\
    11 & 30.4 \\
    12 & 41.7 \\
    13 & 31.6 \\
    14 & 22.4 \\
    15 & 40.4 \\
    16 & 18.1 \\
    17 & 30.7 \\
    18 & 39.5 \\
    19 & 30.6 \\
    20 & 22.7 \\
    21 & 31.2 \\
    22 & 36.4 \\
    23 & 25.7 \\
    24 & 23.1 \\
    25 & 33.6 \\
    26 & 29.3 \\
    27 & 31.1 \\
    28 & 40.4 \\
    29 & 14.6 \\
    30 & 43.0 \\
  }\netinA
  \pgfplotstableread{
    a & b    \\
    1 & 32.1 \\
    2 & 0 \\
    3 & 26.8 \\
    4 & 0 \\
    5 & 26.8 \\
    6 & 0 \\
    7 & 26.7 \\
    8 & 0 \\
    9 & 26.7 \\
    10 & 0 \\
    11 & 26.7 \\
    12 & 0 \\
    13 & 26.8 \\
    14 & 0 \\
    15 & 26.8 \\
    16 & 0 \\
    17 & 26.8 \\
    18 & 0 \\
    19 & 26.8 \\
    20 & 0 \\
    21 & 26.8 \\
    22 & 0 \\
    23 & 26.8 \\
    24 & 0 \\
    25 & 26.7 \\
    26 & 0 \\
    27 & 26.7 \\
    28 & 0 \\
    29 & 26.8 \\
    30 & 0 \\
  }\netinB
  % This should probably go into a file in figures/
  \scalebox{.85}{\begin{tikzpicture}
    \begin{axis}[
        ymin=0,
        legend style={legend pos=south east},
        grid,
        thick,
        ylabel=Net In (kB/s),
        xlabel=Time (min)
      ]
      \addplot[mark=*, blue] table[x=a, y=b]{\netinA};
      \addplot[mark=x, red] table[x=a, y=b]{\netinB};
      \addlegendentry{Net In - Blackbox Exporter};
      \addlegendentry{Net In - Amadeus' Prober};
    \end{axis}
  \end{tikzpicture}}
  \caption[Scrape Analysis]{Scrape Analysis.}\label{fig:scrape-analysis}
\end{figure}

As demonstrated in the \autoref{fig:scrape-analysis}, it is evident that there is no zero traffic in the Blackbox Exporter within 30 minutes, in contrast to Amadeus' Prober, which exhibits regular saw-tooth traffic patterns touching the bottom, indicating periodical drops in traffic to zero. This observation suggests that Prometheus Agent spreads scatters across the time interval, resulting in more optimized and spread network traffic. 