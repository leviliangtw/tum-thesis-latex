\chapter{\abstractname}

%TODO: Abstract

With the increasing complexity and scale of IT infrastructures, Amadeus's traditional monitoring solutions often fall short of providing the necessary observability and operational efficiency. To address this gap, a scalable, reliable, and cloud-ready active monitoring platform becomes a need. The research investigates three potential active monitoring solutions: a self-developed system employing operators, schedulers, and probers; integration of the Prometheus Agent and Blackbox Exporter; and a comprehensive solution using the Prometheus Operator for advanced cloud integration. The study concludes that the Prometheus Operator, with its automated configuration and management capabilities, offers the most alignment with Amadeus's requirements for scalability, reliability, and cloud readiness. It ensures compatibility with custom protocols and optimizes resource usage, notably improving operational efficiency and reducing memory overhead compared to traditional monitoring methods. The architecture integrates seamlessly with Kubernetes and OpenShift, facilitating efficient deployment and management of monitoring components. A detailed evaluation of active monitoring solutions within a complex IT ecosystem was provided to contribute to the broad monitoring community, highlighting the Prometheus Operator's role in enhancing observability and operational efficacy. Future research will focus on automating Blackbox Exporter configurations, developing interfaces for custom probers, integrating eBPF for precise monitoring, and exploring advanced machine learning and tools tailored for microservices, aiming to enhance the efficiency, flexibility, and observability of monitoring solutions in complex IT environments. 